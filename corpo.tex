\chapter{Disposições preliminares}
    \section{Objetivo da fundação}
    A Fundação Asimo é um grupo formado por discentes e docentes de toda a UNIFEI, que desenvolve um trabalho de pesquisa na área de metodologias ativas aplicadas ao STEAM (Ciência Tecnologia Engenharia Artes e Matemática, do Inglês: Science Technology Engineering Art and Math) para o ensino de estudantes das escolas públicas de Itajubá desde o ensino fundamental até o médio, de modo a cobrir um vácuo existente nessas áreas no Brasil e desenvolver as suas habilidades tecnocientíficas. E despertar o interesse pela engenharia nas novas gerações.
    
    \section{Disposições gerais}
    \begin{enumerate}
        \item Realizar estudos e pesquisas, desenvolver tecnologias alternativas, produzir e promover a divulgação de informações, conhecimentos técnicos e científicos de interesse social a fim de minimizar a pobreza e as desigualdades sociais em Itajubá e região;
        \item Criar, coordenar, incentivar, participar e implementar projetos de cunho social, principalmente, vinculados à STEAM;
        \item Promover o treinamento teórico e prático de seus membros voltados para o desenvolvimento humano das comunidades menos favorecidas;
        \item Realizar cursos para a comunidade com o objetivo de promover a transferência de conhecimentos tecnológicos nas áreas de STEAM;
        \item Contribuir na promoção do desenvolvimento econômico e social e no combate a pobreza e desigualdades sociais de Itajubá e região;
        \item Buscar avançar o acesso à educação inclusiva, de qualidade e equitativa, e promover oportunidades de aprendizagem;
        \item Buscar reduzir as desigualdades existentes entre o ensino publico e privado.
        \item Buscar despertar nas novas gerações o interesse pela areá das engenharias.
    \end{enumerate}
 
\chapter{Dos Membros em Geral}  
    \section{Direitos e deveres}
    \begin{enumerate}
        \item Os membros da Fundação Asimo têm o direito de:
        \begin{enumerate}
            \item Receber orientação e supervisão do coordenador do departamento;
            \item Participar em um ou mais departamentos, desde que essa participação não interfira em seus deveres curriculares para com a universidade;
            \item Utilizar os recursos e materiais necessários para o desenvolvimento das atividades do departamento;
        \end{enumerate}
        \item Os membros da Fundação Asimo têm o dever de:
        \begin{enumerate}
            \item Participar das reuniões e atividades do seu departamento;
            \item Comparecer pontualmente às reuniões obrigatórias, demonstrando comprometimento com as atividades da fundação;
            \item Cumprir as tarefas delegadas pelo diretor;
            \item Trabalhar em equipe para atingir os objetivos da fundação;
            \item Manter uma comunicação efetiva com os demais membros e coordenadores, compartilhando informações relevantes para o bom andamento das atividades;
            \item Agir com ética e profissionalismo no desempenho de suas atividades;
            \item Zelar pelos recursos e materiais utilizados nas atividades dentro da fundação;
            \item Respeitar a diversidade de opiniões, experiências e perspectivas dos demais membros, promovendo um ambiente inclusivo e respeitoso;
            \item Manter a confidencialidade de informações sensíveis relacionadas às atividades da fundação, protegendo a integridade e reputação da organização;
            \item Representar a fundação de maneira digna e profissional em eventos externos, quando designado para tal função.
        \end{enumerate}
    \end{enumerate}
 
\chapter{Dos Departamentos}
    \section{Departamento de Docência}
        \subsection{Objetivos}
            O Departamento de Docência da Fundação Asimo tem como objetivo preparar, organizar, administrar, revisar e melhorar as aulas e materiais didáticos utilizados para ministrar cursos, palestras e todo e qualquer evento de ensino de STEAM. Além disso, é responsável por gerar e manter atualizado um banco de dados com as necessidades das instituições parceiras, captar novas instituições para a Fundação, garantir o bom andamento das aulas e elaborar relatórios semestrais dos projetos executados.
            
        \subsection{Responsabilidades}
            \begin{enumerate}
                \item Compete ao Departamento de Docência da Fundação Asimo:
                \begin{enumerate}
                    \item Montar o plano de aula semestral, e segui-lo durante as aulas; 
                    \item Elaborar e revisar constantemente os materiais didáticos utilizados nas aulas;
                    \item Captar novas instituições parceiras para a Fundação;
                    \item Realizar a inscrição dos alunos e membros da Fundação Asimo em eventos;
                    \item Garantir o cumprimento dos planos de atividades, organizando a estrutura de cada proposta de curso, avaliando carga horária, material disponível e afins;
                    \item Elaborar relatórios semestrais dos projetos executados;
                \end{enumerate}
                \item Os membros do Departamento de Docência da Fundação Asimo deverão utilizar vestimenta adequada para dar aulas, tais como calça, tênis e preferencialmente uniforme da Fundação.
            \end{enumerate}
            
        \subsection{Organização}
        \begin{enumerate}
            \item O Departamento de Docência será organizado da seguinte forma:
            \begin{enumerate}
                \item  Diretor de Docência: responsável pelo gerenciamento geral do departamento;
                \item Professores: responsáveis por ministrar aulas, cursos e palestras e pela orientação acadêmica dos alunos;
                \item Auxiliares de Docência: responsáveis pelo suporte às atividades didáticas, como auxílio aos professores na preparação de materiais didáticos, organização de atividades em sala de aula e orientação dos alunos.
            \end{enumerate}
            \item As funções dos membros do departamento de Docência são definidas pelo Diretor de Docência, levando em consideração as habilidades e experiências de cada profissional e as necessidades específicas da instituição de ensino em relação à qualidade do ensino e ao desenvolvimento dos alunos.
        \end{enumerate}
        
        \subsection{Direitos e deveres dos membros}
        \begin{enumerate}
            \item Os membros do Departamento de Docência da Fundação Asimo têm o direito de:
            \begin{enumerate}
                \item Participar de reuniões, treinamentos e capacitações para aprimoramento técnico e profissional;
            \end{enumerate}
            \item São deveres dos membros do Departamento de Docência da Fundação Asimo:
            \begin{enumerate}
                \item Não faltar às aulas que ministram, salvo motivos de saúde e/ou que sejam comunicados previamente com definição de professor substituto sempre que possível, seu descumprimento acarreta duas advertências; 
            \end{enumerate}
        \end{enumerate}
        
        \subsection{Dos Procedimentos}
            \begin{enumerate}
                \item Qualquer solicitação para o uso da sala designada ao projeto deve ser encaminhada à Diretoria Executiva, que avaliará a solicitação.
            \end{enumerate}

    \section{Departamento de Gestão de Pessoas}
        \subsection{Objetivos}
            O departamento de Gestão de Pessoas tem como objetivo promover um ambiente de trabalho saudável, motivador e eficiente, visando ao desenvolvimento contínuo dos colaboradores e à maximização do seu potencial. As atividades desenvolvidas pelo departamento incluem, mas não se limitam a, motivação da equipe, comunicação interna, gestão de desempenho e suporte ao bem-estar dos membros da organização.
            
        \subsection{Responsabilidades}
            Os membros do departamento de Gestão de Pessoas serão responsáveis por:
            \begin{enumerate}
                \item Motivação da Equipe: O Departamento de Gestão de Pessoas é responsável por criar e implementar estratégias que promovam a motivação e o engajamento dos colaboradores, incentivando o reconhecimento, a valorização e o desenvolvimento pessoal e profissional;
                \item Comunicação Interna: Caberá ao departamento facilitar a comunicação interna eficaz entre os membros da organização, garantindo a disseminação clara e transparente de informações relevantes, assim como o estímulo ao diálogo construtivo;
                \item Gestão de Desempenho: O departamento deve desenvolver e implementar sistemas de avaliação de desempenho que permitam identificar as competências individuais, fornecer feedback construtivo e promover planos de desenvolvimento individualizados;
                \item  Suporte ao Bem-Estar: Compete ao departamento promover ações que visem ao bem-estar físico e emocional dos colaboradores, assegurando um ambiente de trabalho saudável, inclusivo e equilibrado.
                \item Documentação Legal: O departamento têm a responsabilidade de elaborar e manter documentos legais que estabeleçam o vínculo entre a Fundação Asimo e suas escolas parceiras. Isso abrange acordos contratuais, termos relacionados à utilização de imagens e gestão de dados, garantindo uma relação transparente e legalmente sólida.
            \end{enumerate}
        
        \subsection{Organização}
        \begin{enumerate}
            \item O Departamento de Gestão de Pessoas será organizado da seguinte forma:
                \begin{enumerate}
                    \item Diretor de Gestão de Pessoas: responsável pelo gerenciamento geral do departamento;
                    \item Equipe de Gestão de Pessoas: membros com habilidades e experiências específicas para desempenhar atividades relacionadas à gestão de recursos humanos, tais como recrutamento, desenvolvimento, monitoramento e controle.
                \end{enumerate}
        \end{enumerate}
        
        \subsection{Direitos e deveres dos membros}   
        \begin{enumerate}
            \item Os membros do Departamento de Gestão de Pessoas da Fundação Asimo têm o direito de:
            \begin{enumerate}
                \item Participar ativamente do planejamento, implementação e avaliação das ações do departamento;
            \end{enumerate}
            \item São deveres dos membros do Departamento de Gestão de Pessoas da Fundação Asimo:
            \begin{enumerate}
                \item Cumprir os planos de atividades elaborados pelo departamento;
                \item Elaborar e manter os documentos que estabeleçam de maneira legal o vínculo entre o projeto e as escolas parceiras, assim como os termos referentes à utilização de imagens e gestão de dados.
            \end{enumerate}
        \end{enumerate}
        
        \subsection{Dos Procedimentos}
            \begin{enumerate}
                \item O departamento realizará periodicamente a avaliação das estratégias e ações implementadas, visando à melhoria contínua dos processos relacionados à gestão de pessoas;
                \item O desenvolvimento de programas de capacitação, treinamento e workshops para os colaboradores será coordenado pelo departamento, em parceria com outras áreas da organização, quando necessário;
                \item As práticas de recrutamento, seleção e integração de novos colaboradores serão definidas e executadas de acordo com os critérios estabelecidos pelo departamento, visando à admissão de profissionais alinhados à cultura e aos valores da organização;
                
            \end{enumerate}
            
   \section{Departamento de Marketing}
        \subsection{Objetivos}
            O departamento de Marketing tem como objetivo principal desenvolver e implementar estratégias de Marketing para a fundação, visando a divulgação das atividades e eventos realizados, bem como a captação de recursos e parcerias. Para isso, serão estabelecidas metas e indicadores para avaliar o desempenho do departamento, com o objetivo de maximizar o impacto das ações de Marketing.
            
        \subsection{Responsabilidades}
            Os membros do departamento de Marketing serão responsáveis por:
            \begin{enumerate}
                \item  Planejar e implementar estratégias de Marketing para a organização, visando a divulgação de suas atividades e eventos, bem como a captação de recursos e parcerias;
                \item Gerenciar as redes sociais e outras plataformas de comunicação da organização, criando conteúdo relevante e promovendo interações com o público;
                \item Desenvolver e gerenciar campanhas publicitárias em diferentes mídias;
                \item Organizar eventos promocionais e institucionais da organização, como palestras, workshops, feiras e congressos;
                \item Gerenciar as parcerias da organização, estabelecendo acordos com empresas e instituições que possam contribuir para seus objetivos;
                \item Monitorar e avaliar os resultados das ações de Marketing, gerando relatórios e análises para a Diretoria e demais departamentos da organização;
                \item Atuar de forma integrada com os demais departamentos da organização, visando a maximização do impacto das ações de Marketing.
                \item Promover a Fundação Asimo em eventos externos ou estabelecer parcerias estratégicas que contribuam para a visibilidade da organização. Essas atividades visam a ampliar o alcance e o impacto das ações de marketing.
            \end{enumerate}
             Além disso, deverão atuar de forma integrada com os demais departamentos da fundação, visando a maximização do impacto das ações de Marketing.
             
        \subsection{Organização}
            \begin{enumerate}
                \item O Departamento de Marketing será organizado da seguinte forma:
                \begin{enumerate}
                    \item Diretor de Marketing: responsável pelo gerenciamento geral do departamento;
                    \item Especialistas em Marketing: responsáveis pela criação e implementação de estratégias de Marketing, incluindo publicidade, pesquisa de mercado, análise de dados e desenvolvimento de campanhas;
                    \item Assistentes de Marketing: responsáveis pelo suporte às atividades de Marketing, como gestão de mídias sociais, produção de conteúdo e colaboração com outras áreas.
                \end{enumerate}
                \item As funções dos membros do departamento de Marketing podem ser definidas pelo diretor de Marketing, levando em consideração as habilidades e experiências de cada profissional e as necessidades específicas do projeto.
            \end{enumerate}
            
        \subsection{Direitos e deveres dos membros}
        \begin{enumerate}
            \item Os membros do Departamento de Marketing têm o direito de:
            \begin{enumerate}
                \item Participar das reuniões do departamento;
                \item Propor ideias para campanhas e ações de Marketing;
                \item Colaborar na execução das campanhas e ações de Marketing;
                \item Receber orientações e treinamentos para aprimorar suas habilidades;
            \end{enumerate} 
            \item  Os membros do Departamento de Marketing têm o dever de:
            \begin{enumerate}
                \item Manter  confidencialidade das informações do departamento;
            \end{enumerate}
        \end{enumerate}
        
        \subsection{Dos Procedimentos}
            \begin{enumerate}
                \item Qualquer membro da Fundação Asimo tem o direito de solicitar a cobertura de marketing para algum evento ou aula. O departamento irá deliberar sobre essa solicitação.
                \item Qualquer membro da Fundação Asimo pode apresentar propostas ao departamento de marketing, as quais serão avaliadas pela equipe.
            \end{enumerate}
            
    \section{Departamento de Projetos}
        \subsection{Objetivos}
        O Departamento de Projetos tem como objetivo principal projetar e executar as demandas dos outros departamentos da Fundação, além de produzir ideias para futuros projetos e melhorar os projetos antigos.
        
        \subsection{Responsabilidades}
        \begin{enumerate}
            \item O Departamento de Projetos é responsável por:
            \begin{enumerate}
                \item Identificar as necessidades dos outros departamentos da Fundação;
                \item Projetar e executar soluções para atender as demandas dos outros departamentos;
                \item Produzir ideias de projetos para o futuro;
                \item Melhorar os projetos antigos, de acordo com as demandas da Fundação;
                \item Coordenar e gerenciar a execução dos projetos.
            \end{enumerate}
        \end{enumerate}
        
        \subsection{Organização}
        \begin{enumerate}
            \item O Departamento de Projetos será organizado da seguinte forma:
            \begin{enumerate}
                \item Diretor de Projetos: responsável pelo gerenciamento geral do departamento;
                \item Equipe de Projetos: membros com habilidades e experiências específicas para realizar atividades de projetos, como planejamento, execução, monitoramento e controle;
                \item Assistentes de Projetos: responsáveis pelo suporte às atividades de gerenciamento de projetos, como coleta de dados, análise de riscos, documentação e comunicação com outras áreas.
            \end{enumerate}
            \item As funções dos membros do departamento de Projetos podem ser definidas pelo gerente de Projetos, levando em consideração as habilidades e experiências de cada membro e as necessidades específicas do projeto.
        \end{enumerate}
        
        \subsection{Direitos e deveres dos membros}
        \begin{enumerate}
            \item Os membros do Departamento de Projetos têm o direito de:
            \begin{enumerate}
                \item Participar das reuniões do departamento;
                \item Propor ideias para projetos;
                \item Colaborar na execução dos projetos;
                \item Receber orientações e treinamentos para aprimorar suas habilidades;
            \end{enumerate} 
            \item  Os membros do Departamento de Projetos têm o dever de:
            \begin{enumerate}
                \item Manter a confidencialidade das informações do departamento;
            \end{enumerate}
        \end{enumerate}
    
        \subsection{Dos Procedimentos}
        \begin{enumerate}
            \item As demandas dos outros departamentos deverão ser encaminhadas por escrito ao Departamento de Projetos, contendo as informações necessárias para a elaboração da proposta de projeto;
            \item O departamento deverá elaborar a proposta de projeto no prazo máximo de 15 dias úteis, a contar do recebimento da demanda;
            \item A proposta de projeto deverá ser submetida à análise e aprovação da Diretoria da Fundação, antes de sua execução;
            \item O departamento deverá manter registros atualizados de todas as atividades relacionadas à execução dos projetos, incluindo relatórios de acompanhamento e avaliação dos resultados;
            \item O departamento deverá apresentar relatórios quinzenais à Diretoria da Fundação, contendo informações sobre o andamento dos projetos e os resultados alcançados a ser anexado às atas das Reuniões de Diretoria Executiva;
            \item Qualquer solicitação para o uso da sala designada ao projeto deve ser encaminhada à Diretoria Executiva, que avaliará a solicitação.  
        \end{enumerate}
         
\chapter{Da Diretoria}
    \section{Composição da Diretoria}
    \begin{enumerate}
       \item A Diretoria será composta da seguinte forma:
       \begin{enumerate}
           \item Presidente
           \begin{enumerate}
               \item Este tendo as seguintes atribuições:
               \begin{enumerate}
               \item  Gerenciar os dados pessoais dos associados;
               \item  Convocar e coordenar as Reuniões Gerais e as reuniões da Diretoria Executiva;
               \item  Nomear procuradores e assessores para fins especiais de representação em juízo ou fora dele;
               \item  Organizar a estrutura funcional da Fundação Asimo;
               \item  Elaborar relatório semestral das atividades realizadas, juntamente com o Departamento de Gestão de Pessoas;
               \item  Emitir certificado para membros com auxílio do professor coordenador da Fundação Asimo, além dos alunos das instituições beneficiadas pelo projeto;
               \item  Buscar o desenvolvimento de parcerias com instituições públicas, privadas e demais estruturas de ordem social, para auxílio na Fundação Asimo.
               \end{enumerate}
           \end{enumerate}
            \item Vice Diretores
                \begin{enumerate}
                    \item Estes tendo as seguintes atribuições:
                    \begin{enumerate}
                        \item Assumir temporariamente as responsabilidades do seu Diretor em ausências.
                        \item Documentar as decisões e ações importantes do seu departamento, bem como os desafios enfrentados e as soluções encontradas.
                    \end{enumerate}
                \end{enumerate}              
           
           \item Diretor de Docência;
           \item Diretor de Gestão de Pessoas;
           \item Diretor de Marketing;
           \item Diretor de Projetos;
           \item Conselheiro.
           \begin{enumerate}
               \item Este tendo as seguintes atribuições:
               \begin{enumerate}
                   \item Fornecer insights experientes para definir metas e direcionar o projeto em alinhamento com os objetivos organizacionais;
                   \item Utilizar contatos valiosos para facilitar parcerias externas e agir como mentor, incentivando o crescimento da equipe diretiva.
               \end{enumerate} 
               \item Para ser conduzido ao cargo de Membro Sênior do projeto, o indivíduo deve:
               \begin{enumerate}
                    \item Ser membro ativo por pelo menos 2 anos no projeto; 
                    \item Ter ocupado cargos na Diretoria do projeto;
                    \item Ser indicado para o cargo pelo professor coordenador, com base no comprometimento e contribuição do mesmo.
               \end{enumerate}
           \end{enumerate}
           
       \end{enumerate}
       \item O termo Diretoria Executiva refere-se ao conjunto de todos os diretores citados.
    \end{enumerate}
    \section{Atribuições da Diretoria}
    \begin{enumerate}
        \item A Diretoria Executiva da Fundação Asimo tem como atribuições:
         \begin{enumerate}
            \item Cumprir e fazer cumprir o regimento interno da Fundação Asimo, as decisões das reuniões e o programa de atividades;
            \item Elaborar o plano de atividades que devem ser apresentados no início de cada semestre;
            \item Sugerir a destituição de qualquer membro da Fundação Asimo, a ser aprovada em Reunião Geral;
            \item Não faltar às Reuniões Gerais, salvo por motivo justificado previamente, e delegando um assessor de sua respectiva Diretoria para substituí-lo;
            \item Assinar o Termo de Compromisso;
            \item Homologar ou negar indicações para a Fundação Asimo, homenagear associados, pessoas físicas e entidades, quando for o caso;
            %\item Se você leu isso envie uma mensagem para o Tony (Presidente) e ganhe um chocolate; 
            \item Convocar Reuniões Gerais extraordinárias quando se fizer necessário;
            \item Participar da reforma do regimento, opinando sobre possíveis mudanças;
            \item Intermediar o contato entre a Fundação Asimo e outros projetos na Universidade Federal de Itajubá;
            \item Manter os arquivos e correspondência da Fundação Asimo em perfeita ordem;
            \item Prestar consultoria para as gestões futuras, quando solicitado;
            \item Manter sob sua responsabilidade o patrimônio financeiro, supervisionando todas as atividades correlacionadas;
            \item Conduzir a contabilidade rigorosamente atualizada;
            \item Propor o orçamento anual;
            \item Coordenar as atividades de captação de recursos do núcleo;
            \item Firmar e administrar patrocínios da Fundação Asimo;
            \item Elaborar e revisar editais que envolvem a captação de recursos;
            \item Comprar materiais necessários para a Fundação Asimo;
            \item Deliberar sobre o uso da sala designada para o projeto.
        \end{enumerate}
    \end{enumerate}
    \section{Deliberações da Diretoria}
        As deliberações da Diretoria devem ser tomadas por maioria simples.
    \section{Renúncia e substituição de membros da Diretoria}
        No caso de renúncia ou destituição de algum membro da Diretoria, a Diretoria Executiva deverá deliberar sobre a convocação de eleições extraordinarias para preenchimento do cargo vago. Caso a proposta não seja aprovada, a Diretoria assumirá a coordenação temporária das atividades do departamento afetado até que ocorra uma eleição subsequente. Essa medida visa garantir a continuidade e estabilidade nas responsabilidades do departamento durante o período de transição.
    \section{Prestação de contas da Diretoria}
        A diretoria deve prestar contas de suas atividades e do patrimônio financeiro da Fundação Asimo na ultima Reunião Geral do semestre.
    \section{Dos Procedimentos}
        Qualquer solicitação para o uso da sala designada ao projeto deve ser encaminhada à Diretoria Executiva, que ira deliberar sobre.
\chapter{Das Reuniões}
    \section{Tipos de reuniões}
    \begin{enumerate}
        \item A Fundação Asimo possui quatro tipos de reuniões:
        \begin{enumerate}
            \item Reuniões Gerais;
            \item Reuniões Internas dos Departamentos;
            \item Reuniões da Diretoria Executiva;
            \item Reuniões Extraordinárias.
        \end{enumerate}
    \end{enumerate}
    
    \section{Convocação das reuniões}
    \begin{enumerate}
        \item Todas as reuniões são programadas no início do semestre podendo ser alteradas mediante a solicitação com antecedência mínima de uma semana;
        \begin{enumerate}
            \item As reuniões gerais são mensais ocorrendo sempre na primeira semana do mês no dia e horário definido no início do semestre;
            \item As reuniões internas dos Departamentos são semanais ocorrendo sempre no mesmo dia e horário;
            \item As reuniões da Diretoria Executiva são semanais ocorrendo sempre no mesmo dia e horário.
        \end{enumerate}
        \item As reuniões extraordinárias são por sua vez divididas em tipos podendo ser convocadas fora do calendário programado no início do semestre: 
        \begin{enumerate}
            \item As reuniões gerais serão convocadas com antecedência mínima de cinco dias úteis;
            \item As reuniões internas dos Departamentos serão convocadas pelos diretores, podendo ser solicitadas por qualquer membro;
            \item As reuniões da Diretoria Executiva serão convocadas com antecedência mínima de dois dias úteis.
        \end{enumerate}
    \end{enumerate}
    \section{Quórum}
        \begin{enumerate}
            \item Para as reuniões gerais, é obrigatória a presença de todos os membros da Fundação Asimo;
            \item Para as reuniões internas dos Departamentos, a presença é obrigatória para o membro daquele Departamento;
            \item Para as reuniões da Diretoria Executiva, é obrigatória a presença de todos os diretores e o(s) membro(s) sênior(s);
            \item Para validar uma reunião, é imprescindível contar com a presença de no mínimo 60$\%$ dos membros obrigatórios, incluindo membros remotos e representantes. Caso essa porcentagem não seja atingida, a reunião será remarcada, ressaltando que a ausência não isenta os membros ausentes de eventuais advertências.
        \end{enumerate}

    \section{Pauta das reuniões}
    \begin{enumerate}
        \item A pauta das reuniões gerais deve ser enviada com a convocação da reunião;
        \item A pauta das reuniões internas dos Departamentos deve ser definida pelos diretores;
        \item A pauta das reuniões da Diretoria Executiva deve ser definida pelo Diretor Geral da Fundação Asimo.
    \end{enumerate}
    
    \section{Deliberações}
    \begin{enumerate}
        \item As deliberações em todas as reuniões devem ser tomadas por maioria simples de votos;
        \item Em caso de empate, caberá ao Diretor Geral da reunião o voto de minerva.
    \end{enumerate}
 
\chapter{Do Ingresso de Novos Membros}
   A admissão de novos membros deverá observar um requisito no qual 70$\%$ da composição total sejam alunos dos cursos oferecidos pelo IESTI (Instituto de Engenharia de Sistemas e Tecnologia da Informação), incluindo ECA (Engenharia de Controle e Automação), ECO (Engenharia da Computação) e ELT (Engenharia Eletrônica).
    \section{Processo seletivo}
    O processo seletivo ocorrerá em três etapas:
    \begin{enumerate}
        \item Análise dos currículos e inscrições;
        \item Entrevista com os candidatos selecionados.
        \item Dinâmica em grupo com os candidatos selecionados.
    \end{enumerate}
    
    \section{Permanência na Fundação}
    \begin{enumerate}
        \item Fica definido prazo máximo de três meses para avaliação dos membros recém aprovados em processo seletivo;
        \item É dever de todos os membros, proporcionar ambiente adequado ao desenvolvimento dos membros;
        \item Os ingressantes podem ser aprovados e tornam-se membros mediante indicação de seu Diretor responsável e aprovação junto à Diretoria Executiva;    
    \end{enumerate}
 
\chapter{Das Eleições}
    \section{Periodicidade das eleições}
    A eleição da Diretoria será realizada a cada semestre da UNIFEI, por meio de voto secreto dos membros da Fundação Asimo.
    \section{Processo eleitoral}
    \begin{enumerate}
        \item Para se candidatar aos cargos de Presidente, Vice-Diretores, Diretor de Marketing, Diretor de Projetos e Diretor de Docência, é necessário:
        \begin{enumerate}
            \item Estar matriculado na UNIFEI;
            \item Ser do departamento especifico;
            \item Passado no período probatório.
        \end{enumerate}
        \item Em caso de empate os criterios de desempate em ordem sao:
        \begin{enumerate}
            \item Mais tempo no projeto;
        \end{enumerate}
        \item O mandato é de um semestre da UNIFEI podendo haver uma única recondução ao cargo;
        \item A votação é feita para cada diretor individualmente, sendo necessária maioria simples.
    \end{enumerate}
 
\chapter{Do Desligamento e Advertências}
    \section{Desligamento voluntário}
        Para o seu desligamento voluntário, o membro da Fundação deverá terminar as suas tarefas
vigentes ou propor uma solução para sua saída.

        
    \section{Advertências}
    \begin{enumerate}
        \item O não cumprimento das atribuições previstas no regimento interno acarretará em uma advertência;
        \item A comunicação de relacionamentos entre membros deve ser reportada ao Departamento de Gestão de Pessoas, caso não seja comunicado acarretara em uma advertência. A ocorrência de problemas envolvendo tais relacionamentos resultará em duas advertências;
        \item O uso inadequado da sala, ou seja, para propósitos que não estejam alinhados com os objetivos do projeto, acarretará em duas ou mais advertências, a ser deliberada pela diretoria executiva;
        \item Faltas sem avisos prévios em reuniões obrigatórias gera uma advertência, para cada falta;
        \item Será aceito 25$\%$ de atraso com tolerância de 10 minutos em Reuniões obrigatórias a cada semestre, o não cumprimento deste acarreta em uma advertência; 
        \item Será aceito 25$\%$ de faltas justificadas em Reuniões obrigatórias a cada semestre, o não cumprimento deste acarreta em uma advertência;
        \item Um diretor que negligenciar suas responsabilidades, como, por exemplo, deixar de comunicar à Diretoria Executiva sobre uma possível advertência a um dos membros sob sua supervisão, estará sujeito a receber duas advertências;
        \item Qualquer membro da fundação poderá indicar uma possível advertência para outro membro, notificando a diretoria executiva, que irá deliberar sobre a indicação;
        \item A advertência tem validade de 6 meses.
    \end{enumerate}
    
    \section{Processo para desligamento}
    \begin{enumerate}
        \item O acúmulo de três advertências acarreta na indicação de desligamento;
        \item A conduta inadequada de um membro enquanto estiver usando a camisa do projeto acarretará na indicação de desligamento;
        \item O processo de desligamento de um membro se concretiza após a sua indicação, a qual será objeto de deliberação pela Diretoria. Se a indicação for aprovada, o membro em questão terá o direito de apresentar uma sustentação oral perante os diretores, os quais, posteriormente, realizarão uma nova deliberação sobre o assunto.
    \end{enumerate}
         
\chapter{Do financeiro e Do Inventario}
    \begin{enumerate}
    \item Dos Procedimentos 
        \begin{enumerate}
            \item Da Aquisição de novos itens
            \begin{enumerate}
                \item A etapa inicial para adquirir itens para o projeto deve ser desencadeada por meio de uma solicitação formal à Diretoria Executiva.
                \item A Diretoria terá um período de cinco dias para avaliar a proposta. 
                \item Se aprovada, a solicitação sera encaminhada para o departamento de gestão para elaboração de um orçamento e uma avaliação da viabilidade econômica da compra.
                \item Com isso o membro encarregado poderá efetuá-la. 
                \item E após a apresentação do comprovante, o presidente, em nome da Diretoria Executiva, realizará o reembolso dos custos relacionados ao item em até cinco dias.
                \item Qualquer aquisição realizada fora dos procedimentos estabelecidos por este regimento será tratada como uma doação ao projeto.
            \end{enumerate}
             
        \end{enumerate}
    \end{enumerate}
 
\chapter{Das Disposições Finais}
    \section{Modificações do regimento interno}
        \begin{enumerate}
            \item Qualquer membro pode propor revisões no documento;
            \item As propostas devem ser encaminhadas à Diretoria Executiva e serão pautadas na próxima Reunião Geral prevista;
            \item Qualquer alteração deverá ser acordada entre no mínimo 50$\%$ mais um legitimada em Reunião Geral.
        \end{enumerate}

        
    \section{Validade do regimento interno}
        O presente Regimento entrará em vigor a partir da Reunião Geral que lhe aprovar e poderá ser revisto, a qualquer momento, em Reunião Geral.
        
    \section{Disposições não previstas no regimento interno}
        Caso ocorram situações não previstas no presente Regimento Interno, as mesmas serão decididas pela Diretoria Executiva, que tomará as medidas cabíveis para solucioná-las da forma mais justa e adequada possível, sempre respeitando os princípios e objetivos da Fundação. Tais decisões deverão ser comunicadas aos membros da Fundação em Reunião Geral.
        As decisões tomadas nestes casos serão registradas em ata e, se necessário, poderão ser incorporadas ao presente Regimento Interno por meio de sua devida alteração e aprovação pelos membros.
